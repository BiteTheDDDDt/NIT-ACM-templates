\documentclass[10pt,a4paper]{article}
%\usepackage{zh_CN-Adobefonts_external}
\usepackage{xeCJK}
\usepackage{amsmath, amsthm}
\usepackage{listings,xcolor}
\usepackage{geometry} % 设置页边距
\usepackage{fontspec}
\usepackage{graphicx}
\usepackage[colorlinks]{hyperref}
\usepackage{setspace}
\usepackage{fancyhdr} % 自定义页眉页脚


\setsansfont{Consolas} % 设置英文字体
\setmonofont[Mapping={}]{Consolas} % 英文引号之类的正常显示,相当于设置英文字体

\linespread{1.2}

\title{Algorithm Templates}
\author{Pxl @ NIT}
\definecolor{dkgreen}{rgb}{0,0.6,0}
\definecolor{gray}{rgb}{0.5,0.5,0.5}
\definecolor{mauve}{rgb}{0.58,0,0.82}

\pagestyle{fancy}

\lhead{\CJKfamily{kai} Ningbo Institute of Technology, Zhejiang University } %以下分别为左中右的页眉和页脚
\chead{}

\rhead{\CJKfamily{kai} 第 \thepage 页}
\lfoot{} 
\cfoot{\thepage}
\rfoot{}
\renewcommand{\headrulewidth}{0.4pt} 
\renewcommand{\footrulewidth}{0.4pt}
%\geometry{left=2.5cm,right=3cm,top=2.5cm,bottom=2.5cm} % 页边距
\geometry{left=3.18cm,right=3.18cm,top=2.54cm,bottom=2.54cm}
\setlength{\columnsep}{30pt}

\makeatletter

\makeatother



\lstset{
    language    = c++,
    numbers     = left,
    numberstyle={                               % 设置行号格式
        \small
        \color{black}
        \fontspec{Consolas}
    },
	commentstyle = \color[RGB]{0,128,0}\bfseries, %代码注释的颜色
	keywordstyle={                              % 设置关键字格式
        \color[RGB]{40,40,255}
        \fontspec{Consolas Bold}
        \bfseries
    },
	stringstyle={                               % 设置字符串格式
        \color[RGB]{128,0,0}
        \fontspec{Consolas}
        \bfseries
    },
	basicstyle={                                % 设置代码格式
        \fontspec{Consolas}
        \small\ttfamily
    },
	emphstyle=\color[RGB]{112,64,160},          % 设置强调字格式
    breaklines=true,                            % 设置自动换行
    tabsize     = 4,
    frame       = single,%主题
    columns     = fullflexible,
    rulesepcolor = \color{red!20!green!20!blue!20}, %设置边框的颜色
    showstringspaces = false, %不显示代码字符串中间的空格标记
	escapeinside={\%*}{*)},
}

\begin{document}
\title{Algorithm Templates}
\author {BiteTheDust}
\maketitle
\tableofcontents
\newpage
\section{动态规划}
\subsection{数位dp}
\lstinputlisting{动态规划/数位dp.cpp}
\subsection{SOSdp}
\lstinputlisting{动态规划/SOSdp.cpp}
\subsection{斜率优化dp}
\lstinputlisting{动态规划/斜率优化dp.cpp}
\subsection{可逆背包}
\lstinputlisting{动态规划/可逆背包.cpp}
\subsection{树形dp}
\lstinputlisting{动态规划/树型dp.cpp}
\subsection{UpDowndp}
\lstinputlisting{动态规划/UpDowndp.cpp}
\subsection{换根dp}
\lstinputlisting{动态规划/换根dp.cpp}
\subsection{基环树dp}
\lstinputlisting{动态规划/基环树dp.cpp}
\section{字符串}
\subsection{Manacher}
\lstinputlisting{字符串/Manacher.cpp}
\subsection{KMP}
\lstinputlisting{字符串/KMP.cpp}
\subsection{EXKMP}
\lstinputlisting{字符串/EXKMP.cpp}
\subsection{字符串hash}
\lstinputlisting{字符串/字符串hash.cpp}
\subsection{后缀自动机}
\lstinputlisting{字符串/后缀自动机.cpp}
\subsection{维护endpos}
\lstinputlisting{字符串/维护endpos.cpp}
\subsection{回文自动机}
\lstinputlisting{字符串/回文自动机.cpp}
\subsection{AC自动机}
\lstinputlisting{字符串/AC自动机.cpp}
\subsection{序列自动机}
\lstinputlisting{字符串/序列自动机.cpp}
\subsection{最小表示法}
\lstinputlisting{字符串/最小表示法.cpp}
\subsection{Lyndon分解}
\lstinputlisting{字符串/Lyndon分解.cpp}
\section{图论}
\subsection{dijkstra}
\lstinputlisting{图论/dijkstra.cpp}
\subsection{dinic}
\lstinputlisting{图论/dinic.cpp}
\subsection{dij费用流}
\lstinputlisting{图论/dij费用流.cpp}
\subsection{zkw费用流}
\lstinputlisting{图论/zkw费用流.cpp}
\subsection{KM算法}
\lstinputlisting{图论/KM算法.cpp}
\subsection{带花树}
\lstinputlisting{图论/带花树.cpp}
\subsection{并查集}
\lstinputlisting{图论/并查集.cpp}
\subsection{割点}
\lstinputlisting{图论/割点.cpp}
\subsection{桥}
\lstinputlisting{图论/桥.cpp}
\subsection{强连通分量}
\lstinputlisting{图论/强连通分量.cpp}
\subsection{树链剖分}
\lstinputlisting{图论/树链剖分.cpp}
\subsection{LinkCutTree}
\lstinputlisting{图论/LinkCutTree.cpp}
\subsection{任意根lca}
\lstinputlisting{图论/任意根lca.cpp}
\subsection{最大独立集}
\lstinputlisting{图论/最大独立集.cpp}
\subsection{拓扑排序}
\lstinputlisting{图论/拓扑排序.cpp}
\subsection{树分治}
\lstinputlisting{图论/树分治.cpp}
\subsection{DsuOnTree}
\lstinputlisting{图论/DsuOnTree.cpp}
\subsection{虚树}
\lstinputlisting{图论/虚树.cpp}
\section{数据结构}
\subsection{莫队分块}
\lstinputlisting{数据结构/莫队分块.cpp}
\subsection{带修改莫队分块}
\lstinputlisting{数据结构/带修改莫队分块.cpp}
\subsection{树状数组}
\lstinputlisting{数据结构/树状数组.cpp}
\subsection{线段树(重载+最大子列和)}
\lstinputlisting{数据结构/线段树(重载+最大子列和).cpp}
\subsection{可持久化线段树}
\lstinputlisting{数据结构/可持久化线段树.cpp}
\subsection{动态开点线段树}
\lstinputlisting{数据结构/动态开点线段树.cpp}
\subsection{李超树}
\lstinputlisting{数据结构/李超树.cpp}
\subsection{整体二分}
\lstinputlisting{数据结构/整体二分.cpp}
\subsection{扩展整体二分}
\lstinputlisting{数据结构/扩展整体二分.cpp}
\subsection{字典树}
\lstinputlisting{数据结构/字典树.cpp}
\subsection{块状链表}
\lstinputlisting{数据结构/块状链表.cpp}
\section{数论}
\subsection{GCD、LCM、EXGCD}
\lstinputlisting{数论/GCD、LCM、EXGCD.cpp}
\subsection{乘法逆元}
\lstinputlisting{数论/乘法逆元.cpp}
\subsection{MillerRobin、PollardRho}
\lstinputlisting{数论/MillerRobin、PollardRho.cpp}
\subsection{快速幂乘}
\lstinputlisting{数论/快速幂乘.cpp}
\subsection{矩阵快速幂}
\lstinputlisting{数论/矩阵快速幂.cpp}
\subsection{快速阶乘}
\lstinputlisting{数论/快速阶乘.cpp}
\subsection{欧拉函数}
\lstinputlisting{数论/欧拉函数.cpp}
\subsection{欧拉降幂}
\lstinputlisting{数论/欧拉降幂.cpp}
\subsection{线性基}
\lstinputlisting{数论/线性基.cpp}
\subsection{线性筛}
\lstinputlisting{数论/线性筛.cpp}
\subsection{杜教筛}
\lstinputlisting{数论/杜教筛.cpp}
\subsection{组合数}
\lstinputlisting{数论/组合数.cpp}
\subsection{模系解码、分数间最小分子}
\lstinputlisting{数论/模系解码、分数间最小分子.cpp}
\subsection{EXCRT}
\lstinputlisting{数论/EXCRT.cpp}
\subsection{n次同余}
\lstinputlisting{数论/n次同余.cpp}
\subsection{离散对数}
\lstinputlisting{数论/离散对数.cpp}
\subsection{BerlekampMassey、ReedsSloane}
\lstinputlisting{数论/BerlekampMassey、ReedsSloane.cpp}
\subsection{拉格朗日插值法}
\lstinputlisting{数论/拉格朗日插值法.cpp}
\subsection{高斯消元}
\lstinputlisting{数论/高斯消元.cpp}
\subsection{Dirichlet}
\lstinputlisting{数论/Dirichlet.cpp}
\subsection{类欧几里德}
\lstinputlisting{数论/类欧几里德.cpp}
\section{其他}
\subsection{01分数规划}
\lstinputlisting{其他/01分数规划.cpp}
\subsection{单纯形法}
\lstinputlisting{其他/单纯形法.cpp}
\subsection{java大数}
\lstinputlisting{其他/java大数.cpp}
\subsection{mini读入挂}
\lstinputlisting{其他/mini读入挂.cpp}
\subsection{FFT快速傅里叶变换}
\lstinputlisting{其他/FFT快速傅里叶变换.cpp}
\subsection{FWT快速沃尔什变换}
\lstinputlisting{其他/FWT快速沃尔什变换.cpp}
\subsection{NTT快速数论变换}
\lstinputlisting{其他/NTT快速数论变换.cpp}
\subsection{MTT快速数论变换(任意模数)}
\lstinputlisting{其他/MTT快速数论变换(任意模数).cpp}
\subsection{SG函数}
\lstinputlisting{其他/SG函数.cpp}
\subsection{第二类斯特林数}
\lstinputlisting{其他/第二类斯特林数.cpp}
\subsection{离散化}
\lstinputlisting{其他/离散化.cpp}
\subsection{STL}
\lstinputlisting{其他/STL.cpp}
\subsection{星期}
\lstinputlisting{其他/星期.cpp}
\subsection{博弈}
\lstinputlisting{其他/博弈.cpp}
\section{几何}
\subsection{二维几何}
\lstinputlisting{几何/二维几何.cpp}
\subsection{三维几何}
\lstinputlisting{几何/三维几何.cpp}
\subsection{三维凸包}
\lstinputlisting{几何/三维凸包.cpp}
\subsection{多边形交}
\lstinputlisting{几何/多边形交.cpp}
\section{工具}
\subsection{对拍}
\lstinputlisting{工具/对拍.cpp}
\subsection{数据生成器}
\lstinputlisting{工具/数据生成器.cpp}
\subsection{cb环境配置}
\lstinputlisting{工具/cb环境配置.cpp}
\end{document}